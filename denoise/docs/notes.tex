% Created 2020-04-10 Fri 18:41
\documentclass[11pt]{article}
\usepackage[utf8]{inputenc}
\usepackage[T1]{fontenc}
\usepackage{fixltx2e}
\usepackage{graphicx}
\usepackage{longtable}
\usepackage{float}
\usepackage{wrapfig}
\usepackage{rotating}
\usepackage[normalem]{ulem}
\usepackage{amsmath}
\usepackage{textcomp}
\usepackage{marvosym}
\usepackage{wasysym}
\usepackage{amssymb}
\usepackage{hyperref}
\tolerance=1000
\author{Kapil}
\date{\today}
\title{notes}
\hypersetup{
  pdfkeywords={},
  pdfsubject={},
  pdfcreator={Emacs 25.2.2 (Org mode 8.2.10)}}
\begin{document}

\maketitle
\tableofcontents

\section{GO Label/Scoring Strategy}
\label{sec-1}
The set of labels I am using for training and validation comes from
the human set of annotations for GO. I'm pretty sure my generated
FuncAssociate file will do the job.

GO annotations relate genes -> annotations
GO ontology relates annotations -> annotations

FuncAssociate file maps GO annotations -> set of genes 

The difficulty is that each gene has several GO annotations available
to it. A simple binary hit/miss will not suffice. Instead, the scoring
strategy will rank the labels and pick the highest ranked label.

\subsection{Label Strategy}
\label{sec-1-1}
Let $G = (V, E)$ be our graph. Let $\sigma$ be our set of labels. Then
a function $L : \Sigma -> V$ defines the set of labels for a node
$v \in V$ as $L^{-1}(v)$.

\subsection{Scoring Strategy}
\label{sec-1-2}
Suppose an unlabeled gene A had neighbors B, C, and D where

B is annotated with 1 and 2
C is annotated with 1, 2, and 3
D is annotated with 2

Then in a standard majority vote scheme we would predict A to have
labels 2, 1 and 3 in descending rank.

\section{Network Enhancement Port}
\label{sec-2}
The network enhancement algorithm takes an input graph as a matrix
$W$ and clears out all nodes with no neighbors. Then it creates a
diffusion matrix $P = D^-1W$. As this diffusion matrix is directed, we
take the mean of both directed weights as the undireced weight. That
is we then set $W = 1/2(P + P^T)$
\section{Notes}
\label{sec-3}
We are using the link prediction embedding to add new links and then
do FP on the new network.
% Emacs 25.2.2 (Org mode 8.2.10)
\end{document}