% Created 2020-04-24 Fri 18:59
% Intended LaTeX compiler: pdflatex
\documentclass[11pt]{article}
\usepackage[utf8]{inputenc}
\usepackage[T1]{fontenc}
\usepackage{graphicx}
\usepackage{grffile}
\usepackage{longtable}
\usepackage{wrapfig}
\usepackage{rotating}
\usepackage[normalem]{ulem}
\usepackage{amsmath}
\usepackage{textcomp}
\usepackage{amssymb}
\usepackage{capt-of}
\usepackage{hyperref}
\usepackage{minted}
\usepackage[margin=1in]{geometry}
\author{Henri Schmidt}
\date{\today}
\title{}
\hypersetup{
 pdfauthor={Henri Schmidt},
 pdftitle={},
 pdfkeywords={},
 pdfsubject={},
 pdfcreator={Emacs 26.3 (Org mode 9.1.9)}, 
 pdflang={English}}
\begin{document}

\tableofcontents


\section{Setup}
\label{sec:org5e38917}
This experiment tested the results of running two function prediction
methods (WMV and KNN after embedding) on the DREAM3 network after
performing denoising on it by adding edges using link prediction.

Two link prediction methods were used. Normalized DSD created an
embedding and then pairwise distance was used to rank the edges. The
GLIDE method was ran as specified by Kapil with the parameters 
set as follows:

\begin{minted}[mathescape, 
               xleftmargin=2pt, 
               xrightmargin=2pt, 
               style=autumn, 
               framesep=3mm,
               frame=lines
               ]{python}
params = {"alpha" : 1, "beta" : 1000, "delta" : 0.001, "loc" : "l3"}

def foo(a):
  pass
\end{minted}

For both link prediction methods, the top 10\% of edges were added to
the network with a weight of 1.

Performance was evaluated using 5-fold CV with the GO label set from
2019 where all GO labels annotating less than 50 or greater than 1000
proteins are discared (I use Lily's code to do this).

\section{Results}
\label{sec:org2ca20e5}
The accuracy for each method is listed below:

\begin{verbatim}
Majority Vote: 
WMV on raw network: 11.804767309875142%
WMV on DSD added edges network: 11.804767309875142%
WMV on GLIDE added edges network: 11.804767309875142%

10-Nearest Neighbors using DSD embedding:
KNN on raw network: 12.553916004540294%
KNN on DSD added edges network: 12.57661748013621%
KNN on GLIDE added edges network: 12.57661748013621%
\end{verbatim}

\section{Conclusion}
\label{sec:org18ebe93}
It is clear that when using WMV there was no difference after
denoising the network by adding edges. I have a few ideas explaining
this behaviour. First, we add edges with a weight of \(1\), while the
raw network contains very high confidence interactions with weight up
to \(105\). I believe that these high weight interactions have the final
say because of their weight and that creating fake low weight
interactions makes little difference. Second, the GO label set is
hierarchical, so I imagine that more popular terms are always being
predicted. Terms more specific to certain proteins aren't counted for
as much.

It is good to see that performance either stays the same or improves
using both denoising and both FP methods. However, I am a little
concerned that the local aspect of GLIDE is not being captured as it
carries the \textbf{exact} same importance as normalized DSD. I think some
parameter tweaking is definitely necessary, and the GLIDE code
reviewed.
\end{document}